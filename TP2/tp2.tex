\documentclass{article}
\usepackage[utf8]{inputenc}

\title{TP2}
\author{Thibault Duhamel 18026048, Heng Shi 18171434 }

\usepackage{natbib}
\usepackage{graphicx}
\usepackage{amsmath}

\begin{document}

\maketitle

\section*{Question 1}

The loss function is defined as follows: 

$$E(\overrightarrow{w}) = \sum_n^N{(t_n - \overrightarrow{w}^T.\overrightarrow{\phi_n})^2} + \lambda \overrightarrow{w}^T.\overrightarrow{w}$$

As we want to minimize the above expression, it is convenient to calculate its gradient:

$$\frac{dE(\overrightarrow{w})}{d\overrightarrow{w}} = \sum_n^N{\big[-2(t_n - \overrightarrow{w}^T.\overrightarrow{\phi_n})\overrightarrow{\phi_n}^T\big]} + 2\lambda \overrightarrow{w}$$

Let us now set this gradient to zero in order to deduce $\overrightarrow{w}$:

$$\frac{dE(\overrightarrow{w})}{d\overrightarrow{w}} = 0$$

$$\sum_n^N{\big[-2(t_n - \overrightarrow{w}^T.\overrightarrow{\phi_n})\overrightarrow{\phi_n}^T\big]} + 2\lambda \overrightarrow{w} = 0$$

$$\sum_n^N{\big[ (\overrightarrow{w}^T.\overrightarrow{\phi_n})\overrightarrow{\phi_n}^T\big]} - \sum_n^N{\big[t_n\overrightarrow{\phi_n}^T\big]} + \lambda \overrightarrow{w} = 0$$

We can write the exact same equality using matrices, with $\Phi$ being the same matrix as defined in the book from Bishop:

$$ \Phi^T \Phi \overrightarrow{w} - \Phi^T \overrightarrow{t}  + \lambda \overrightarrow{w} = 0$$

$$ \Phi^T \Phi \overrightarrow{w} + \lambda \overrightarrow{w} = \Phi^T \overrightarrow{t}$$

$$ (\Phi^T \Phi + \lambda I)\overrightarrow{w} = \Phi^T \overrightarrow{t}$$

And finally, assuming the matrix $(\Phi^T \Phi + \lambda I)$ is invertible:

$$\boxed{\overrightarrow{w} = (\Phi^T \Phi + \lambda I)^{-1}\Phi^T \overrightarrow{t}}$$

\section*{Question 2}

In this question, we do not write the transpose symbol to simplify expressions.

\bigskip

The cross-entropy loss function is defined as follows:

$$ E(\overrightarrow{w}) = -\sum_n^N{\big[t_n ln(\sigma(\overrightarrow{w}.\overrightarrow{\phi_n})) + (1-t_n)ln(1-\sigma(\overrightarrow{w}.\overrightarrow{\phi_n}))\big]} $$

Let us first compute the gradient of the sigmoid function: 

$$\frac{d\sigma}{d\overrightarrow{w}} = \frac{\overrightarrow{\phi_n}e^{-\overrightarrow{w}.\overrightarrow{\phi_n}}}{(1+e^{-\overrightarrow{w}.\overrightarrow{\phi_n}})^2}$$

Then, we can use this intermediate result to calculate the gradient of the 2 logarithms in the sum:

$$\frac{dln(\sigma)}{d\overrightarrow{w}} =
\frac{\frac{d\sigma}{d\overrightarrow{w}}}{\sigma} $$

$$= \frac{\overrightarrow{\phi_n}e^{-\overrightarrow{w}.\overrightarrow{\phi_n}}}{(1+e^{-\overrightarrow{w}.\overrightarrow{\phi_n}})^2} \times (1+e^{-\overrightarrow{w}.\overrightarrow{\phi_n}}) $$ 

$$=
\frac{\overrightarrow{\phi_n}e^{-\overrightarrow{w}.\overrightarrow{\phi_n}}}{1+e^{-\overrightarrow{w}.\overrightarrow{\phi_n}}}
$$

and:

$$\frac{dln(1 - \sigma)}{d\overrightarrow{w}} =
\frac{-\frac{d\sigma}{d\overrightarrow{w}}}{1 - \sigma}$$

$$=\frac{-\overrightarrow{\phi_n}e^{-\overrightarrow{w}.\overrightarrow{\phi_n}}}{(1+e^{-\overrightarrow{w}.\overrightarrow{\phi_n}})^2} \times \frac{1+e^{-\overrightarrow{w}.\overrightarrow{\phi_n}}}{e^{-\overrightarrow{w}.\overrightarrow{\phi_n}}}$$

$$=\frac{-\overrightarrow{\phi_n}}{1+e^{-\overrightarrow{w}.\overrightarrow{\phi_n}}}$$

From those 2 expressions, we are now able to express the gradient of the whole sum:

$$\frac{dE(\overrightarrow{w})}{d\overrightarrow{w}} = - \sum_n^N{\Bigg[ t_n \frac{\overrightarrow{\phi_n}e^{-\overrightarrow{w}.\overrightarrow{\phi_n}}}{1+e^{-\overrightarrow{w}.\overrightarrow{\phi_n}}} + (1-t_n) \frac{-\overrightarrow{\phi_n}}{1+e^{-\overrightarrow{w}.\overrightarrow{\phi_n}}} \Bigg]}$$

$$= - \sum_n^N{\Bigg[ t_n \frac{e^{-\overrightarrow{w}.\overrightarrow{\phi_n}}}{1+e^{-\overrightarrow{w}.\overrightarrow{\phi_n}}} + (1-t_n) \frac{-1}{1+e^{-\overrightarrow{w}.\overrightarrow{\phi_n}}} \Bigg]}\overrightarrow{\phi_n}$$

$$= - \sum_n^N{\Big[ t_n y_n e^{-\overrightarrow{w}.\overrightarrow{\phi_n}} + (t_n-1) y_n \Big]}\overrightarrow{\phi_n}$$

$$= - \sum_n^N{\Big[ t_n y_n (1+e^{-\overrightarrow{w}.\overrightarrow{\phi_n}}) - y_n \Big]}\overrightarrow{\phi_n}$$

$$= - \sum_n^N{\Big[ t_n y_n \frac{1}{y_n} - y_n \Big]}\overrightarrow{\phi_n}$$

$$= - \sum_n^N{\big[ t_n - y_n \big]}\overrightarrow{\phi_n}$$

$$\boxed{\frac{dE(\overrightarrow{w})}{d\overrightarrow{w}} = \sum_n^N{\big[ y_n - t_n \big]}\overrightarrow{\phi_n}}$$

\section*{Question 3}

Let us first write down the definition of the entropy in this context:

$$H(X) = - p_1log_2(p_1) - p_2log_2(p_2) - p_3log_2(p_3)$$

As we want to maximize this function (with $p_1+p_2+p_3 = 1$) we can express the problem with a Lagrange multiplier:

$$L = p_1log_2(p_1) + p_2log_2(p_2) + p_3log_2(p_3) + \lambda(p_1+p_2+p_3-1)$$

However, we do have another condition on those probabilities (that is $p_1 = 2p_2$), so we can modify the expression consequently:

$$L = 2p_2log_2(2p_2) + p_2log_2(p_2) + p_3log_2(p_3) + \lambda(3p_2+p_3-1)$$

$$L = 2p_2 + 2p_2log_2(p_2) + p_2log_2(p_2) + p_3log_2(p_3) + \lambda(3p_2+p_3-1)$$

$$L = 2p_2 + 3p_2log_2(p_2) + p_3log_2(p_3) + \lambda(3p_2+p_3-1)$$

Let us now compute the gradients of such a function:

$$
\begin{cases}
\frac{dL}{dp_2} = 2 + 3log_2(p_2) + \frac{3}{ln(2)} + 3\lambda$$\\
\frac{dL}{dp_3} = log_2(p_3) + \frac{1}{ln(2)} + \lambda$$
\end{cases}
$$

\bigskip

We can then set those gradients to zero:

$$ 
\begin{cases}
2 + 3log_2(p_2) + \frac{3}{ln(2)} + 3\lambda = 0\\
log_2(p_3) + \frac{1}{ln(2)} + \lambda = 0
\end{cases}
$$

Let us first divide the first equation by 3:

$$ 
\begin{cases}
\frac{2}{3} + log_2(p_2) + \frac{1}{ln(2)} + \lambda = 0\\
log_2(p_3) + \frac{1}{ln(2)} + \lambda = 0
\end{cases}
$$

As both equations contain the same terms, let us substract $L_2$ from $L_1$:

$$log_2(\frac{p_3}{p_2}) = \frac{2}{3}$$

It is now possible to extract a relation between $p_3$ and $p_2$:

$$ p_3 = 2^{2/3} p_2$$

As all probabilities now depend on $p_2$, we can insert those terms inside the condition $p_1 + p_2 + p_3 = 1$:

$$2 p_2 + p_2 + 2^{2/3}p_2 = 1$$

Which gives:

$$\boxed{p_2 = \frac{1}{2^{2/3}+3}}$$

Now that we know $p_2$, it is eventually trivial to compute $p_1$ and $p_3$:

$$\boxed{p_1 = \frac{2}{2^{2/3}+3}\ \text{and}\ p_3 = \frac{2^{2/3}}{2^{2/3}+3}}$$

\end{document}

